\documentclass[../annexe.tex]{subfiles}
\graphicspath{{\subfix{ressources}}}

\begin{document}



\subsection{Application 1}



\subsubsection{Afficher l'historique des taux de change}

\textbf{Acteurs :} \\
Le client et le serveur. \\

\textbf{Description :} \\
Permet au client de consulter l'historique des taux de change pour chaque monnaie. \\

\textbf{Précondition :} \\
La monnaie est supportée par l'application. \\

\textbf{Fréquence :} \\
Souvent pour les utilisateurs internationaux. \\

\textbf{Parcours de base :} \\
\begin{enumerate}
    \item Le client se dirige sur la bonne fenêtre
    \item Il entre une monnaie de base et une monnaie destination
    \item L'application communique au serveur via l'API afin de récuperer dans la base de donnée les informations nécessaires
    \item L'application trace le graphique
\end{enumerate}
\bigskip

\textbf{Postconditions :} \\
L'application affiche l'historique du taux de change. \\



\subsubsection{Caluler un taux de change}

\textbf{Acteurs :} \\
Le client et le serveur. \\

\textbf{Description :} \\
Permet au client d'afficher et calculer un taux de change. \\

\textbf{Précondition :} \\
La monnaie est supportée par l'application. \\

\textbf{Fréquence :} \\
Souvent pour les utilisateurs internationaux. \\

\textbf{Parcours de base :} \\
\begin{enumerate}
    \item Le client se dirige sur la bonne fenêtre
    \item Il entre une monnaie de base, un montant de base et une monnaie destination
    \item L'application communique au serveur via l'API afin de récuperer dans la base de donnée les informations nécessaires
    \item L'application affiche le taux utilisé et calcule la valeur finale dans la monnaie destination
\end{enumerate}
\bigskip

\textbf{Postconditions :} \\
L'application affiche la valeur finale et le taux utilisé. \\



\subsection{Application 2}



\subsubsection{Définir des frais de transaction}

\textbf{Acteurs :} \\
L'institution et le serveur. \\

\textbf{Description :} \\
Permet à une institution de définir ses frais de transactions internationaux. \\

\textbf{Fréquence :} \\
Une fois lorsque l'intitution configure son système. \\

\textbf{Parcours de base :} \\
\begin{enumerate}
    \item L'institution se dirige sur la bonne fenêtre
    \item Elle entre un pays destination et un taux
    \item L'application communique au serveur via l'API afin d'écrire dans la base de données le taux
    \item L'application affiche dans un tableau les taux déjà définis
\end{enumerate}
\bigskip

\textbf{Postconditions :} \\
L'application affiche le taux fraichement défini dans le tableau. \\



\end{document}
