\documentclass[../rapport.tex]{subfiles}

\begin{document}

\subsection{Design et inspiration:}
    
    L'idée derrière l'interface graphique de l'application client était d'avoir un GUI simple et épuré comme il se fait maintenant.
    Pour ça nous nous sommes basée sur Google non seulement pour le thème blanc (même si on laisse le choix de thème a l'utilisateur);
    mais aussi pour le design des boutons et la police utilisée. Le but était d'avoir une interface graphique simple et minimalise qui 
    Collais bien avec le projet. Une application bancaire doit être le plus clair et concis possible pour permettre aux utilisateurs d'accéder
    rapidement à leurs informations.

\subsection{Outils:}

    Ayant décidé de faire les différentes scènes en fxml et de gérer les interactions avec des contrôleurs, nous avons beaucoup utilisé 
    SceneBuilder qui a grandement facilité la création des différents fichiers fxml. Le fait de pouvoir visualiser directement à quoi ressemblait
    une scène sans avoir besoin de compiler et d'exécuter le fichier mais aussi le fait d'avoir une interdace facile d'utilisation qui nous 
    permettait de facilement ajouter des éléments aux scènes nous a vraiment aidés. Le seul problème que nous avons eu avec SceneBuilder est que
    le rendu qu'il affiche ne correspondait pas toujours à ce qu'on avait une fois la scène affichée sur nos écrans.

\subsection{Différences avec les schémas:}

    Les différences avec le modèle fournist au premier quadrimestre ne sont pas nombreuses hormis le fait que d'autres scènes qui modelèrent 
    Ne figuraient pas dans les schémas ont été rajoutés par exemple la scène des paramètres. Ceci s'explique notamment par le fait que 
    durant la conception nous n'avions envisagé que le plus gros des scènes et ce qui directement en relation avec les fonctionnalités
    de l'application.


\subsection{Manques à corriger:}
    Nous n'avons malheureusement pas pu peaufiner notre interface graphique. Il reste encore des bugs aux lancements comme des textures 
    qui n'est pas appliquées ou encore des labels qui ne changent pas de tails en fonction du texte lors d'un changement de langue.
    \newpage
\end{document}