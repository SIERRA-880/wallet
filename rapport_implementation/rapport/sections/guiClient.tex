\documentclass[../rapport.tex]{subfiles}

\begin{document}

\subsection{Design et inspiration:}
    
    L'idée derrière l'interface graphique de l'application client était d'avoir un GUI simple et épuré comme il se fait maintenant.
    Pour ça nous nous sommes basé sur Google non seulement pour le thème blanc (même si on laisse le choix de thème a l'utilisateur);
    mais aussi pour le design des boutons et la police utilisée. Le but était d'avoir une interface graphique simple et minimalise qui 
    Collait bien avec le projet. Une application bancaire doit être le plus clair et concis possible pour permettre aux utilisateurs d'accéder
    rapidement à leurs informations.

\subsection{Outils:}

    Ayant décidé de faire les différentes scènes en fxml et de gérer les interactions avec des controlleurs, nous avons beaucoup utilisé 
    SceneBuilder qui a grandement facilité la création des différents fichiers fxml. Le fait de pouvoir visualiser directement à quoi ressemblera
    une scène sans avoir besoin de compiler et d'exécuter le fichier mais aussi le fait d'avoir une interface facile d'utilisation qui nous 
    permettait de facilement ajouter des éléments aux scènes nous a vraiment aidés. Le seul problème que nous avons eu avec SceneBuilder est que
    le rendu qu'il affiche ne correspond pas toujours à ce qu'on obtient une fois dans l'application finale.

\subsection{Différences par rapport à la modélisation:}

    Les différences avec le modèle fournit au premier quadrimestre ne sont pas nombreuses hormis le fait que de nouvelles scènes qui 
    ne figuraient pas dans les schémas ont été rajoutées par exemple la scène des paramètres. Ceci s'explique notamment par le fait que 
    durant la conception nous n'avions envisagé que le plus gros des scènes dont celles étant directement en relation avec les fonctionnalités de base
    de l'application. Certaines adaptations ont donc été faites.


\subsection{Manques à corriger:}
    Nous n'avons malheureusement pas pu peaufiner notre interface graphique. Il reste encore des bugs 
    tel que  des labels ne changeant pas de taille dynamiquement en fonction du texte lors d'un changement de langue.\\
    De plus certains éléments graphiques sont manquants et empêchent par conscéquent l'utilisatation de certaines fonctionnalités comme expliqué dans la description de l'application.
    \newpage
\end{document}
