\documentclass[../rapport.tex]{subfiles}

\begin{document}
Le code de l'application client est divisé en 3 packages : APP, API et GUI.
En plus de ces packages, il existe la classe App qui est la classe principale de l'application. 

\subsection{Le package APP}
Ce package contient les objets de l'application. Ces objets doivent pouvoir lire des jsons venant de l'api et inversement ils doivent être capable
d'écrire leurs données sous forme d'un json. Dans ce but, il existe une classe Data pour chaque objet de l'application. Cette classe permet de stocker 
les données de l'objet et d'y accéder via les différentes méthodes get et set. 
Afin de remplir les classes Data avec des données reçues de l'api, chaque classe d'objet posède une méthode read\_data capable d'associer
chaque donnée d'un json à l'attribut correspondant de la classe Data gràce à la librairie Jackson.  
Certaine classes possèdent une méthode write\_data qui effectue le travail inverse : elle retourne un json contenant les données de la classe Data.
Par rapport à la phase de modélisation, peu de choses ont été changée dans l'implémentation. Certaines classes ont été ajoutée mais elles suivent le même schéma que 
les autres classes modélisées.

\begin{enumerate}
\item Account -> compte de paiment
\item User -> l'utilisateur
\end{enumerate}

\subsection{Le package API}
La classe Api de ce package contient toutes les méthodes permettant d'effectuer les requêtes à l'api nécessaires au fonctionnement de l'application client.
Le package HttpClient est utilisé pour envoyer les requêtes de type GET POST et PUT.
La classe JsonReader est une interface implémentée par les classes capables de lire un json
La class JsonTools possède la méthode splitJson permettant de séparer plusieurs json contenu dans une seule chaine de caractère.

\subsection{Le package GUI}
Le package GUI contient les controllers où sont définies les méthodes de l'interface graphique. 
Dans tous les controllers se trouve une méthode initialize utilisée pour placer les données reçues de l'api dans l'interface graphique et à changer l'interface
selon la langue courante choisie par l'uitilisateur.

\subsection{Gestion des langues}
Par rapport à la phase de modélisation, peu de choses ont été changée dans l'implémentation. Certaines classes ont été ajoutée 
Le language par defaut de l'application est l'anglais mais il peut être changé par un utilisateur. Dans les paramètres de l'application, 
l'utilisateur peut choisir la langue qu'il souhaite utiliser. Son choix est alors enregistré dans la base de données et lorsqu'il se reconnectera,
le language choisi sera automatiquement utilisé. De plus, le dernier language utilisé est affiché lorsque l'application est lancée même si l'utilisateur 
ne s'est pas encore authentifié.

\subsection{Fonctionnalités non-implémentées ou à améliorer}
Suite à un retard au niveau de l'implementation, certaines fonctionnalités ne sont pas complètes ou non implémentées.
La gestion des languages permet à l'utilisateur d'ajouter ses propres langues en ajoutant un fichier json dans les ressources de l'application. Cependant, ce nouveau
languages n'est pas disponible via le GUI. En ajoutant les les propositions de languages en fonction des fichiers présents dans les ressources. 
Lors de la réalisation d'un virement, il était demander d'ajouter une étape de vérification pendant laquelle l'utilisateur doit entrer un pin afin de confirmer la transaction.
Celà n'a pas été implémenté. L'écran de connexion aurait pu être réutilisé en ajoutant une méthode de vérification de pin.
L'exportation et importations des données n'est pas implémentée, toutes les données sont accessibles mais aucune méthode ne permet de créer le fichier csv.
Les comptes d'épargne ne sont pas disponibles dans l'application. 
\newpage
\end{document}
