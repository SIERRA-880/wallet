\documentclass[../rapport.tex]{subfiles}

\begin{document}

\subsection{Design et inspiration:}
    
L'idée derrière l'interface graphique de l'application 2 est la même que pour l'application 1 avec un petit changement qui est le fait 
que nous avons voulu implémenter des fonctionnalités qui se déclencheraient avec un click gauche.


\subsection{Outils:}

    Comme pour la première application, l'outil scene builder a été utilisé afin de créer les fichiers fxml de chaque scenes.

\subsection{Différences avec les schémas:}

    Les différences avec le modèle fournit au premier quadrimestre sont le fait que nous avons ajouté des scènes.\\
    Celles ayant déjà été schématisées 
    ont été gardées et sont presques identiques, seules quelques modifications d'apparence ont été effectuées.


\subsection{Manques à corriger:}

    Comme décrit dans la section design, le plus gros problème rencontré avec l'application 2 était l'implémentation du click gauche qui
    aurait du faire apparaitre un sous-menu. Nous avons d'abord essayé de créer les sous-menus et de les afficher en récupérant les coordonnées de 
    l'endroit où se trouvait la souris au moment du click et d'y faire apparaitre le sous-menu. Malheureusement on n'arrivait pas à 
    soit faire apparaitre le menu au bon endroit, soit le faire disparaitre une fois un autre click effectué.\\  
    On se retrouvait donc
    avec un écran rempli de sous-menus. La solution fut trouvée bien trop tard pour l'implémenter. Il suffisait, d'utiliser la classe
    ContextMenu qui était une classe spécialement dédiée à la création de sous-menus. Les difficultées pour implémenter fidèlement cette partie du gui par rapport à la modélisation créèrent du retard au niveau de la réalisation de cette application en plus du retard initial.\\
    Nous n'avons pu explorer cette piste plus en profondeur ce qui rend le GUI de l'application 2 incomplet.

    \newpage
\end{document}
