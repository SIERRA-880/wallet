\documentclass[../rapport.tex]{subfiles}

\begin{document}
Durant la phase d'implémentation du projet, il était essentiel de configurer un outil de CD/CI afin de détecter tous problème lors de chaque commit.
Étant plusieurs à travailler sur le projet, une erreur pouvait vite arriver et ne pas être détectée à temps, compromettant le travail futur des autres développeurs. 

Nous avons décidé d'utiliser Github actions qui s'intègre parfaitement avec notre repository hébergé sur le même site.
Lors de chaque commit sur les branches contenant les applications frontend et backend, Github actions va faire un build avec Gradle, effectuer les tests et build la javadoc. 

Pour les applications frontend, Github actions va également générer 2 fatjars, un pour UNIX et un pour Windows.
Une fois générés, ils sont automatiquement ajouté dans une release afin d'être facilement téléchargés.

Pour l'application backend (l'API), un fatjar est également créé.
Celui ci est automatiquement envoyé avec la commande scp sur un serveur personnel et l'API est redémarrée.

Ces opérations prennent environ 3 minutes pour être réalisées.
Une fois terminées, nous recevons une notification sur un serveur Discord qui nous permet de savoir si tout s'est bien passé ou si il y a eu des erreurs.
\end{document}
