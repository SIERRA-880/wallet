\documentclass[../rapport.tex]{subfiles}

\begin{document}



\subsubsection{Créer un compte}

\textbf{Acteurs :} \\
Créer un compte. \\

\textbf{Description :} \\
Permet au client de créer un compte avec un mail et un mot de passe pour pouvoir utiliser l’application. \\

\textbf{Précondition :} \\
L’application est lancée et aucun compte n’est connecté. \\

\textbf{Fréquence :} \\
Chaque première fois qu’un nouveaux client lance l’application. \\

\textbf{Parcours de base :} \\
\begin{enumerate}
    \item Le client clique sur l'option "créer un compte"
    \item Le client entre le mail et le mot de passe qui ser associé au compte
    \item L'application emvoie les identifiants au serveur pour vérification
    \item Le serveur confirme que le mail n'est pas déjà pris
    \item Le srveur crée le compte dans la base de données
    \item Le serveur connecte automatiquement le nouveau client
    \item Le client accède au reste de l'application avec le compte fraichement créé
\end{enumerate}
\bigskip

\textbf{Parcours alternatif :}
\begin{enumerate}
    \item Le serveur retourne que le mail est déjà associé à un compte existant. L'application le signale au client et le renvoie à l'étape 2
\end{enumerate}

\textbf{Postconditions :} \\
Le client est connecté à l'application. \\



\subsubsection{Se connecter}

\textbf{Acteurs :} \\
Le client et le serveur. \\

\textbf{Description :} \\
Permet aux clients de s'identifier et d'accéder à l'application ainsi que toutes ses fonctionnalités. \\

\textbf{Précondition :} \\
L'application est lancée et le client n'est pas déjà connecté à l'pplication. \\

\textbf{Fréquence :} \\
A chaque lancement de l'application. \\

\textbf{Parcours de base :} \\
\begin{enumerate}
    \item Le client entre ses données d'utilisateur (numéro registre national + mot de passe)
    \item L'applicaton envoie les données au serveur pour la vérification
    \item Le serveur confirme que les données sont correctes
    \item Le client peut accéder aux autres parties de l'application
\end{enumerate}
\bigskip

\textbf{Parcours alternatif :}
\begin{enumerate}
    \item Les données sont incorrectes, on retourne à l'étape 1
\end{enumerate}

\textbf{Postconditions :} \\
Le GUI affiche désormais les différents portefeuilles de l'utilisateur stockés dans la base de donnée. \\



\subsubsection{Créer un portefeuilles}

\textbf{Acteurs :} \\
 Le client et le serveur. \\

\textbf{Description :} \\
Création d'un nouveau portefeuille vide avec le créateur comme gestionnaire et le seul membre du nouveau portefeuille. \\

\textbf{Précondition :} \\
Pas de préconditions. \\

\textbf{Fréquence :} \\
Utilisation peur fréquente. \\

\textbf{Parcours de base :} \\
\begin{enumerate}
    \item Le client sélectionne l'option de créer un nouveau portefeuille
\end{enumerate}
\bigskip

\textbf{Parcours alternatif :}
\begin{enumerate}
    \item 
    \item 
\end{enumerate}

\textbf{Postconditions :} \\
. \\



\subsubsection{}

\textbf{Acteurs :} \\
. \\

\textbf{Description :} \\
. \\

\textbf{Précondition :} \\
. \\

\textbf{Fréquence :} \\
. \\

\textbf{Parcours de base :} \\
\begin{enumerate}
    \item 
    \item 
\end{enumerate}
\bigskip

\textbf{Parcours alternatif :}
\begin{enumerate}
    \item 
    \item 
\end{enumerate}

\textbf{Postconditions :} \\
. \\



\subsubsection{}

\textbf{Acteurs :} \\
. \\

\textbf{Description :} \\
. \\

\textbf{Précondition :} \\
. \\

\textbf{Fréquence :} \\
. \\

\textbf{Parcours de base :} \\
\begin{enumerate}
    \item 
    \item 
\end{enumerate}
\bigskip

\textbf{Parcours alternatif :}
\begin{enumerate}
    \item 
    \item 
\end{enumerate}

\textbf{Postconditions :} \\
. \\



\subsubsection{}

\textbf{Acteurs :} \\
. \\

\textbf{Description :} \\
. \\

\textbf{Précondition :} \\
. \\

\textbf{Fréquence :} \\
. \\

\textbf{Parcours de base :} \\
\begin{enumerate}
    \item 
    \item 
\end{enumerate}
\bigskip

\textbf{Parcours alternatif :}
\begin{enumerate}
    \item 
    \item 
\end{enumerate}

\textbf{Postconditions :} \\
. \\



\subsubsection{}

\textbf{Acteurs :} \\
. \\

\textbf{Description :} \\
. \\

\textbf{Précondition :} \\
. \\

\textbf{Fréquence :} \\
. \\

\textbf{Parcours de base :} \\
\begin{enumerate}
    \item 
    \item 
\end{enumerate}
\bigskip

\textbf{Parcours alternatif :}
\begin{enumerate}
    \item 
    \item 
\end{enumerate}

\textbf{Postconditions :} \\
. \\



\subsubsection{}

\textbf{Acteurs :} \\
. \\

\textbf{Description :} \\
. \\

\textbf{Précondition :} \\
. \\

\textbf{Fréquence :} \\
. \\

\textbf{Parcours de base :} \\
\begin{enumerate}
    \item 
    \item 
\end{enumerate}
\bigskip

\textbf{Parcours alternatif :}
\begin{enumerate}
    \item 
    \item 
\end{enumerate}

\textbf{Postconditions :} \\
. \\



\subsubsection{}

\textbf{Acteurs :} \\
. \\

\textbf{Description :} \\
. \\

\textbf{Précondition :} \\
. \\

\textbf{Fréquence :} \\
. \\

\textbf{Parcours de base :} \\
\begin{enumerate}
    \item 
    \item 
\end{enumerate}
\bigskip

\textbf{Parcours alternatif :}
\begin{enumerate}
    \item 
    \item 
\end{enumerate}

\textbf{Postconditions :} \\
. \\



\subsubsection{}

\textbf{Acteurs :} \\
. \\

\textbf{Description :} \\
. \\

\textbf{Précondition :} \\
. \\

\textbf{Fréquence :} \\
. \\

\textbf{Parcours de base :} \\
\begin{enumerate}
    \item 
    \item 
\end{enumerate}
\bigskip

\textbf{Parcours alternatif :}
\begin{enumerate}
    \item 
    \item 
\end{enumerate}

\textbf{Postconditions :} \\
. \\



\subsubsection{}

\textbf{Acteurs :} \\
. \\

\textbf{Description :} \\
. \\

\textbf{Précondition :} \\
. \\

\textbf{Fréquence :} \\
. \\

\textbf{Parcours de base :} \\
\begin{enumerate}
    \item 
    \item 
\end{enumerate}
\bigskip

\textbf{Parcours alternatif :}
\begin{enumerate}
    \item 
    \item 
\end{enumerate}

\textbf{Postconditions :} \\
. \\

\newpage
\end{document}
