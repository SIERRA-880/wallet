\documentclass[../rapport.tex]{subfiles}

\begin{document}

\subsubsection{Ré-initialiser le mot de passe}

\textbf{Acteurs :} \\
Le client ou l'utilisateur et le serveur. \\

\textbf{Description :} \\
Permet à une personne qui a oublié son mot de passe ou qui souhaite le changer de le réinitialiser. \\

\textbf{Précondition :} \\
La personne (client ou institution) est dans l’écran de connection ou dans ses paramètres du compte. \\

\textbf{Fréquence :} \\
Assez rare \\

\textbf{Parcours de base :} \\
\begin{enumerate}
    \item La personne clique sur l’option de ré-initialisation de mot de passe
    \item La personne entre le mail du compte dont elle veut changer le mot de passe
    \item L’application envoie le mail au serveur pour vérifier s’il existe un compte associé à l’adresse mail
    \item Le compte est trouvé et un mail est envoyé avec un lien pour entrer un nouveau mot de passe
    \item La personne entre 2 fois son nouveau mot de passe
    \item Le serveur change le mot de passe du compte dans la base de données
\end{enumerate}
\bigskip

\textbf{Parcours alternatif :} \\
\begin{enumerate}
    \item Aucun compte avec ce mail n'est trouvé, message d'erreur et retour à l'étape 2.
    \item La personne n'entre pas 2 fois le même mot de passe, retour à l'étape 5.
\end{enumerate}
\bigskip

\textbf{Postconditions :} \\
Le mot de passe pour le compte associé au mail donné est changé par le nouveaux mot de passe entré par la personne.



\subsubsection{Se déconnecter}

\textbf{Acteurs :} \\
Le client et le serveur. \\

\textbf{Description :} \\
Permet au client de se déconnecter de l'application. \\

\textbf{Précondition :} \\
Le client est connecté à un compte. \\

\textbf{Fréquence :} \\
De temps en temps. \\

\textbf{Parcours de base :} \\
\begin{enumerate}
    \item Le client demande à l'application de le déconnecter
    \item L'application détruit l'objet utilisateur représentant le client authentifié
    \item L'application ramène le client sur l'écran de connexion
\end{enumerate}
\bigskip

\textbf{Postconditions :} \\
Le client est déconnecté et de retour à l'écran de connexion. Il n'a plus accès à ses données ni au reste de l'application.



\subsubsection{Changer de langue}

\textbf{Acteurs :} \\
Le client et le serveur. \\

\textbf{Description :} \\
Change la langue d'affichage et l'application. \\

\textbf{Précondition :} \\
L'application est lancée. \\

\textbf{Fréquence :} \\
Peu fréquent. \\

\textbf{Parcours de base :} \\
\begin{enumerate}
    \item Le client sélectionne l'option de changement de langue
    \item L'application affiche les langues disponibles
    \item Le client sélectionne la langue désirée
    \item L'application affiche un avertissement et demande une confirmation
    \item Le client confirme le changement de la langue
    \item L'application envoie une requête au serveur pour chnager la langue d'affichnage du client
\end{enumerate}
\bigskip

\textbf{Parcours alternatif :}
\begin{enumerate}
    \item Le client annule le changement de la langue, l'application renvoie sur le GUI les paramètres
    \item Une erreur de connexion a eu lieu et la requête a échoué, pas de changement de la langue, l'utilisateur est averti.
\end{enumerate}

\textbf{Postconditions :} \\
La langue d'affichage de l'application est changée.

\newpage
\end{document}
