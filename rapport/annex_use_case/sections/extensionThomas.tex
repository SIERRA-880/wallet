\documentclass[../annexe.tex]{subfiles}
\graphicspath{{\subfix{ressources}}}

\begin{document}

\subsection{Use-cases de l'application 1}

\subsubsection{Demander un devis pour un type d'assurance}

\paragraph{Acteurs :} Le client et le serveur 


\paragraph{Descrition :} Permet au client de demander un devis pour un certain type d'assurance.


\paragraph{Préconditions :} Le client se trouve sur l'écran lié à la demande de devis.


\paragraph{Fréquence :} De temps en temps (lorsque le client désire un devis).



\paragraph{Parcours de base :}

	\begin{enumerate}
	
		\item Le client clique sur l'option de demande d'un devis pour une assurance.
		\item Le client choisit l'assurance propre à l'institution pour laquelle il désire avoir un devis.
		\item L'application envoie les informations au serveur.
		\item Le serveur renvoie le devis.
		\item Le client consulte le devis.
	\end{enumerate}
	
\paragraph{Postconditions :} La GUI affiche le devis que le client a demandé.

\newpage

\subsubsection{Obtenir des informations sur les différents types 
d'assurances}

\paragraph{Acteurs :} Le client et le serveur 

\paragraph{Description :} Permet au client d'obtenir des informations sur toutes les assurances que propose l'insitution financière à laquelle est lié le portefeuille sélectionné.

\paragraph{Préconditions :} Le client doit être connecté, avoir sélectionné un  de ses portefeuilles être entré dans la gestion des produits financiers.

\paragraph{Fréquence :} De temps en temps.

\paragraph{Parcours de base :}

	\begin{enumerate}
		\item Le client clique sur la fenêtre liée aux informations concernant les assurances.
		\item L'application envoie la demande d'informations au serveur avec l'institution pour laquelle elles sont demandées.
		\item Le serveur renvoie la liste des assurances de l'institution ainsi que tous les détails les concernants.
		\item Le client consulte les informations.
	\end{enumerate}
	
\paragraph{Postconditions :} La GUI affiche la liste des assurances pour l'institution sélectionnée ainsi que toutes les informations qui en découlent.

\newpage

\subsubsection{Accéder à la liste des assurances :}

\paragraph{Acteurs :} Le client et le serveur.

\paragraph{Description :} Un client accède à la liste des assurances auxquelles il a soucrit dans l'institution qui correspond au portefeuille sélectionné.

\paragraph{Préconditions :} Le client doit avoir choisi un de ses portefeuille auquel il voulait accéder et doit avoir choisi de gérer ses produits financiers.

\paragraph{Fréquence :} Assez souvent

\paragraph{Parcours de base :} 

	\begin{enumerate}
		\item Le client clique sur l'accès à ses assurances.
		\item L'application envoie au serveur une demande de récupération des assurances pour le client en particulier.
		\item Le serveur renvoie les diverses assurances auxquelles le client a soucrit s'il en a. 
		Si pas le serveur ne renvoie rien.
		\item Le client consulte ses assurances.
	\end{enumerate}

\paragraph{Postconditions :} La GUI affiche la liste des assurances du client et la possibilité d'en rajouter.

\paragraph{Points d'extensions :} 
\begin{enumerate}
	\item Résilier une assurance
	\item Souscrire à une assurance
\end{enumerate}

\paragraph{Parcours alternatif :}
		Le client coche la case permettant d'afficher également les assurances qu'il a résilié. Et le serveur renvoie également les assurances qui sont désactivées 
		afin de les afficher.
\newpage

\subsubsection{Gérer les paramètres :}

\paragraph{Acteurs :} Le client et le serveur.

\paragraph{Description :} Le client accède aux paramètres d'une des assurances auxquelles il a souscrit. 

\paragraph{Préconditions :} Le client doit avoir sélectionné une des assurances présentes dans sa liste d'assurances.

\paragraph{Fréquence :} Peu souvent

\paragraph{Parcours de base :}

	\begin{enumerate}
		\item Le client sélectionne une de ses assurances.
		\item L'application envoie une demande au serveur pour recevoir les informations liées à cette assurance.
		\item Le serveur renvoie les informations de cette assurance.
		\item Le client clique sur le bouton des paramètres pour cette assurance.
		\item L'application affiche l'onglet des paramètres pour assurance.
		\item Le client effectue des modifications.
		\item L'application envoie les modifications au serveur.
		\item Le serveur effectue les modifications dans la base de donnée.
		\item Le serveur envoie les nouvelles informations à l'application.
		\item L'application affiche les nouvelles informations.
		\item Le client quitte les paramètres.
	\end{enumerate}

\paragraph{Postconditions :} La GUI affiche l'assurance avec les nouveaux paramètres qui lui ont été appliqués.

\paragraph{Parcours alternatif :} Le client n'effectue pas de modifications et quitte la fenêtre.

\newpage

\subsubsection{Visualiser l'historique :}

\paragraph{Acteurs :} Le client et le serveur 

\paragraph{Description :} Le client accède à l'historique concernant ses assurances. Il peut y voir ses paiements et ses retraits.

\paragraph{Préconditions :} Le client doit se trouver dans la liste des assurances.

\paragraph{Fréquence :} Peu souvent 

\paragraph{Parcours de base :} 

	\begin{enumerate}
		\item Le client clique sur le bouton d'affichage de l'historique dans la fenêtre d'une assurance.
		\item L'application envoie la demande d'historique spécifique au client au serveur.
		\item Le serveur renvoie les informations liées à l'historique.
		\item L'application affiche l'historique de l'assurance du client
		\item Le client consulte son historique 
	\end{enumerate}

\paragraph{Postconditions :} La GUI affiche l'historique de l'assurance du client.

\newpage

\subsubsection{Souscrire à une assurance :}

\paragraph{Acteurs :}

		Le cleint et le serveur.

\paragraph{Description :}
		
		Le client décide depuis la liste de ses assurances de sopuscrire à une nouvelle assurance dans l'insitution à laquelle sont portefeuille est lié.

\paragraph{Préconditions :}
		Le client doit posséder un portefeuille dans l'institution en question et se trouver dans la liste de ses assurances.
\paragraph{Fréquence :}
		Quelques fois ( lorsque le client n'a pas encore d'assurances ) et peu souvent lorsqu'il en a déjà.
\paragraph{Parcours de base :}
		\begin{enumerate}
				\item Le client clique sur le bouton d'ajouter d'une assurance.
				\item Le client est rediriger dans la fenêtre de souscription à une assurance.
				\item Le client est amené à choisir l'assurance à laquelle il souhaite souscrire. 
				\item Le client sélectionne le compte avec lequel il souhaite effectuer le paiement de la prime.
				\item L'application envoie les informations de la souscription et du paiement au serveur.
				\item Le serveur vérifie que le client possède suffisamment de liquidités sur le compte sélectionné.
				\item Le serveur effectue le débit du compte et met à jour la prochaine date de paiement de l'assurance.
				\item Le serveur ajoute l'inscription à la liste des assurances du client en question.
				\item Le serveur envoie la confirmation à l'application ainsi que les informations liées à l'assurance.
				\item L'application renvoie le client sur la scène de la liste des assurances et affiche la nouvelle assurance dans la liste.
		\end{enumerate}
\paragraph{Postconditions :}
		La GUI affiche la nouvelle assurance dans la liste des paiements ainsi que la prochaine date de paiement. La GUI met également à jour l'affichage du compte qui a
		été débité dans la liste des comptes du client.

\paragraph{Parcours alternatif :} A partir de la vérification serveur.
\begin{enumerate}
		\item Le serveur détecte qu'il n'y a pas assez d'argent sur le compte sélectionné par le client.
		\item Le serveur annule l'ajout de l'assurance.
		\item Le serveur renvoie un message d'erreur à l'application stipulant que le solde du compte est insuffisant.
		\item L'application affiche l'erreur et demande au client de sélectionner un nouveau compte ou d'annuler se requête.
		\item retour au parcours de base où le client choisit un compte.
\end{enumerate}
\newpage

\subsubsection{Résiliez une assurance :}

\paragraph{Acteurs :}
		Le client et le serveur.
\paragraph{Description :}
		Le client souhaite résiliez une assurance à laquelle il avait souscrit.
\paragraph{Préconditions :}
		Le client doit est à au minimum 3 mois de la prochaine date de paiement de l'assurance sans quoi il devra payer la prime de celle-ci avant de pouvoir la résiliez.
\paragraph{Fréquence :}
		Peu souvent.
\paragraph{Parcours de base :}
		\begin{enumerate}
				\item Le client sélectionne l'assurance et dans le menu de celle-ci clique sur le bouton de suppression.
				\item L'application demande une confirmation au client.
				\item Le client effectue son choix, s'il annule il a ramené sur la visualisation de son assurance
				\item L'application envoie au serveur la requête de suppression de l'assurance en question.
				\item Le serveur contrôle la date de paiement de l'assurance.
				\item Le serveur envoie la confirmation que la date est conforme aux critères de résiliation et désactive l'assurance dans la base de donnée.
				\item Le serveur envoie la confirmation de suppression à l'application.
				\item L'application renvoie le client sur la liste des assurances et lui confirme la résiliation.
				\item Le client peut à présent consulter ses assurances et voir que son assurance est bien désactivée.
		\end{enumerate}
\paragraph{Postconditions :}
		La GUI affiche bien que l'assurance est désactivée si le client affiche les assurance désactivées et n'affiche plus son assurance sans cette option.

\paragraph{Parcours alternatif :}
\begin{enumerate}
		\item Le serveur contrôle le temps restant avant le paiement restant de l'assurance.
		\item Le serveur envoie une erreur à l'application stipulant que le client doit payer la prime de cette dernière car il est trop tard pour la résilier.
		\item L'application affiche le message d'erreur et renvoie l'utilisateur sur le menu de l'assurance.
\end{enumerate}
\newpage

\subsubsection{Payer la prime :}
		
\paragraph{Acteurs :}
		Le client et le serveur.
\paragraph{Description :}
		Le client effectuer le paiement de sa prime que celui-ci soit de manière automatique ou bien manuelle.
\paragraph{Préconditions :}
		Il doit se trouver dans le menu de l'assurance en question.
\paragraph{Fréquence :}
		A chaque renouvèlement de l'assurance.
\paragraph{Parcours de base :}
		\begin{enumerate}
				\item Le client clique sur la bouton pay dans le menu de l'assurance ou a sélectionné au préalable le paiement automatique.
				\item Le client sélectionne le compte avec lequel il souhaite effectuer le paiement ou dans le cas d'un paiement automatique l'avait sélectionné
						lors de l'activation de la feature.
				\item L'application envoie les informations de paiement au serveur.
				\item Le serveur 
		\end{enumerate
\paragraph{Postconditions :}

\newpage

\subsubsection{Payer la primer :}

\paragraph{Acteurs :}

\paragraph{Description :}

\paragraph{Préconditions :}

\paragraph{Fréquence :}

\paragraph{Parcours de base :}

\paragraph{Postconditions :}

\newpage

\subsubsection{Verser/retirer de l'argent d'une assurance :}

\paragraph{Acteurs :}

\paragraph{Description :}

\paragraph{Préconditions :}

\paragraph{Fréquence :}

\paragraph{Parcours de base :}

\paragraph{Postconditions :}

\subsection{Use-case Application 2}


\end{document}
