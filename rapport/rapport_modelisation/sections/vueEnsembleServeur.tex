\documentclass[../rapport.tex]{subfiles}

\begin{document}

On pourrait croire que le serveur n'a pas une place importante dans notre modélisation car il n'est pas beaucoup représenté.
Pourtant il a joué un rôle crucial dans la reflexion du projet car il est un peu l'élement central qui définit la logique des applications. \\
Ces dernières communiquent avec lui via une API de type REST qui sera implémentée en Java et hébergée sur AlwaysData. 
Nous retrouvons également la base de données représentée par l'Entity Relation Diagram sur le serveur. \\
\\
Pour des raisons évidentes de sécurité, le serveur est le seul à pouvoir gérer la base de temps. Cela empêche tout problème du à un pc n'étant pas à l'heure et qui pourrait fausser l'heure d'execution d'une transaction.
Nous utilisons le temps au format UNIX time qui est stockable dans un simple int facilitant grandement les éventuelles opérations à faire.


\end{document}
